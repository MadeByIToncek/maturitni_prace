\documentclass[12pt, a4]{article}
\renewcommand{\baselinestretch}{1.5} 
\usepackage[a4paper,left=30mm,top=25mm,right=30mm,bottom=25mm]{geometry}
\usepackage[utf8]{inputenc}
\usepackage[T1]{fontenc}
\usepackage{hyperref}
\usepackage{url}
\usepackage{csquotes}
\usepackage[czech]{babel}
\fontfamily{ptm}

\author{Antonín Vřešťál}
\title{Tvorba astronomických videí}

\begin{document}
\maketitle
Téma tvorby astronomických videí jsem si vybral z důvodu mého dlouhodobého zapojení v projektu AstroCrew. AstroCrew je skupina mladých nadšenců pro astronomii, která se věnuje tvorbě popularizačních astronomických videí shrnujících aktuální dění na obloze pro širokou veřejnost. Mojí částí práce v tomto projektu je spojování grafické, textové a zvukové složky do výsledného videa. 

V teoretické části této projektové maturitní práce podrobně popíši tvorbu astronomického videa z technické stránky, přípravu animací pro simulační program Stellarium (program pro simulaci noční oblohy), organizaci práce ve skupině a sjednocení všech vstupů do výsledného videa. Součástí celého procesu je neustálé kontrolování možných chyb, které se do finálního videa mohou dostat v každém kroku. Tyto kontrolní mechanismy budou v maturitní práci také popsány. Specifickou výzvou tohoto projektu je vytvoření obsahu, který je možné znovu použít každý rok ve stejném období bez nutnosti aktualizace obsahu. Důležitou součástí teoretické části práce bude také podrobný popis postupného vývoje tvorby videí od prvních nesmělých pokusů pro planetárium v iQLandii až po relativně profesionální video vytvořené v rámci praktické části této maturitní práce.

Jako praktickou část této projektové maturitní práce jsem si zvolil tvorbu videa o situaci na noční obloze v měsíci prosinec. Důvodem, proč jsem si zvolil měsíc prosinec, je skutečnost, že video s tímto tématem ještě nebylo v rámci projektu AstroCrew vytvořeno. Ve videu se zaměříme na souhvězdí vycházející zimní oblohy, objekty hlubokého vesmíru (mlhoviny, hvězdokupy, galaxie, atd.), meteorický roj Geminid a jasné hvězdy, které na obloze můžeme v tuto roční dobu pozorovat. Naopak do videa nezahrneme planety, možnost polárních září ani komety, jejichž přítomnost na obloze je v rámci let proměnlivá.
\end{document}